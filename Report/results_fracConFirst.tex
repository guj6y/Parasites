\begin{subfigure}[cb!]{.45\textwidth}
\caption{Concomittant no refuge:}
\makeFigureAllSizes{frac-con-first}{3}{Frac. cons. extct first}
\end{subfigure}
\begin{subfigure}[cb!]{.45\textwidth}
\caption{Refuge and Concomittant:}
\makeFigureAllSizes{frac-con-first}{4}{}
\end{subfigure}

\caption{This figure gives the fraction of links in which the consumer went extinct before the resource.  This is a fraction of the total number of links that were removed from the web.}
\label{fig:fracConFirst}
\end{figure}
Note that in figure \ref{fig:fracConFirst}, we probably expect the value to be a bit more than .5 since basal species almost never go extinct.  
This suggests that consumers generally drive their prey to extinction (top-down effect?).  
As the fraction of parasites increases, the the fraction of consumers extinct first increases, indicating that parasites tend to starve before their prey (though we can see this more clearly in another plot).  
There is no clear difference between the four models.  
The value when consumers have body size ratio equal to 10 is somewhat higher, which suggests that these smaller consumers tend to exert slightly more pressure on their prey.  
This makes sense since their metabolic rates will be on average only 1.8 ($10^{-.25}$) times smaller than their prey, as compared to 3.4 ($100^{-.25}$
) times smaller when the consumer resource body size ratio is 100.
The previous statement applies most cleanly to the null case with no parasites. 
As we add parasites, more and more resources go extinct before consumers (though the effect is still small).  Smaller parasites also generally raise the fraction of consumers extinct before their prey.  Again this probalby makes sense since they will end up having much higher metabolic rates; it is difficult for them to find enough food to satisfy their high metabolic requirements.  
